% Chapter 1

\chapter{Clinical Background} % Main chapter title

\label{chapter1} % For referencing the chapter elsewhere, use \ref{Chapter1}


\section{Importance of Biomarkers in ICU Studies}

Allocation of resources to patients to minimize mortality is a constant priority for healthcare professionals. This is especially important in the area of healthcare we have chosen to focus on in this paper: critical care,  where resources such as equipment and attention of specialists are  scarce. A critical care specialist focuses on the most vulnerable and urgent patients who are placed in an Intensive Care Unit (ICU). In this setting, the specialist is often faced with a difficult decision of which patients to allocate resources to.
The COVID-19 global pandemic has recently put such a predicament under the spotlight. A recent study on ICU capacity in Wuhan, the COVID-19's epicentre in China, states that at a point during the current crisis the number of COVID-19 patients who need ICU resources was 1120. However, only 600 ICU beds existed. As a result, out of the patients who died, only 25\% who required intubation and mechanical ventilation had received it \citep{wu2020characteristics}. Another study conducted in Lombardy, COVID-19's epicentre in Italy, states that under pre-crisis conditions,  the city's  total ICU capacity of 720 already operates at 85\% - 90\% occupancy during winter months. To make things worse, during the first two weeks after the city's first confirmed COVID-19 case, the number of COVID-19 related ICU admissions rose exponentially to 556. Moreover, the exponential estimates for total number of cases in Lombardy projected the tally to increase to 14,452 within two weeks, suggesting an impending and ever more drastic disparity between supply and demand of ICU resources \citep{grasselli2020}. In other words, ICU capacities that are stressed under normal conditions become further strained during times of crises. Faced with such a perpetual dilemma of resource allocation, a critical care specialist uses various biomarkers (physiological indicators) to try and predict severe outcomes in the patients cohort. As such, the discovery and analysis of connections between various biomarkers and unfavourable patient outcomes is a persistent priority in medical research\citep{ware2017biomarkers}. 

\section{Problem: PF Ratio is an Important but Challenging Biomarker}

One such biomarker that is tracked in ICU settings is the  \PF(PF ratio). The numerator – \Pa – refers to the partial pressure of oxygen in arterial blood. It is measured in mmHg via drawing a sample of blood from an artery in the wrist or groin, and analysing it in the laboratory. The denominator – \Fi – refers to the initial fraction of inspired oxygen and is approximately 21\% in breathable atmospheric air. The PF ratio is used to monitor the patient's pulmonary functions. In an ICU, it can be controlled by providing the patient with oxygen concentrations above 21\% using devices such as a mechanical ventilator. It is most notably used in the diagnosis of fatal respiratory illnesses such as Acute Respiratory Distress Syndrome (ARDS) \citep{bernard1994american}. Moreover, it has been shown to be a significant identifer of mortality risk in the general ICU population \citep{villar2011risk} as well as in specific subsets of patients such as newborns with Meconium Aspiration Syndrome (MAS)  \citep{narayanan2019pao2} and post-operation cardiac surgery patients \citep{esteve2014evaluation}.

Despite the various merits of the PF ratio in medical diagnoses and patient monitoring, there is a major challenge associated with its use – the measurement of its numerator, \Pa. The \Pa measurement procedure is invasive and delayed, making it impractical to track \Pa at frequent intervals for all patients. 


\section{Motivation: Can SF Ratio be an Alternative to PF Ratio and for whom?}
A biomarker that is more convenient to measure than \Pa is \Sp or peripheral capillary oxygen saturation. It is defined as the ratio of oxygenated haemoglobin to the total amount of haemoglobin in the blood. It is measured using a pulse oximeter, which uses the principle that oxygenated and deoxygenated haemoglobin absorb and radiate particular wavelengths of light to different extents \citep{jubran1999pulse} . The oximeter illuminates light at specific wavelengths through the skin (usually at the fingertips) and almost instantaneously calculates the ratio of absorption of these wavelengths to extrapolate the proportion of oxygenated haemoglobin in the blood, or \Sp \citep{jubran2015}. Therefore, unlike \Pa, \Sp can be measured in a non-invasive and instantaneous manner. 

Using \Sp instead of \Pa in the calculation of the PF ratio gives a different biomarker ratio - \SF (SF ratio). Although the SF ratio might seem as an intuitive replacement to the \PF, a critical difference between \Pa and \Sp is that the former is generally a more accurate measure of a patient's oxygenation level; it is measured directly from a main artery while the latter is measured at the end of capillaries. 

Nevertheless, several studies have linked \Sp to mortality in patients with certain conditions. For instance, the Tromsø study in 2015 concluded that an \Sp $\leq95\%$ is associated with all-cause mortality and mortality caused by pulmonary diseases after adjusting for sex, age, history of smoking, self-reported diseases and respiratory symptoms, BMI, and CRP concentration \citep{vold2015low}. Another study established \Sp as a predictor of mortality in patients with systemic sclerosis  \citep{swigris2009exercise}. 

Furthermore, SF ratio, which includes \Sp, is linked to mortality in certain patient populations. For instance, SpO\textsubscript{2}/\Fi Time-at-Risk (SF-TAR), defined as the total time spent with severe hypoxemia (SF ratio $\leq 145$), is  significantly correlated with hospital mortality for mechanically ventilated patients, and is as good or a better predictor of it than arterial gas-derived measurements of the PF ratio. Moreover, SF ratio is a non-invasive surrogate for the PF ratio to diagnose certain patient populations such as children with ALI or ARDS \citep{rice2007comparison} and children with smoke inhalation injury \citep{cambiaso2017correlation}. Hence, both SF ratio and \Sp have been shown to be significant predictors of mortality in certain subpopulations. 

However, not all patient subpopulation exhibit a significant link between \Sp or SF ratio and mortality. For example, a prospectively planned meta-analysis study using participant data from 5 randomized clinical trials (conducted from 2005-2014) of infants born before 28 weeks' gestation period found no significant difference between a lower \Sp target range (85\%-89\%) and a higher \Sp target range (91\%-95\%) on mortality or major disability at a corrected age of 18 to 24 months \citep{askie2018association}. Therefore, it seems that the use of \Sp or SF ratio as a predictor of mortality might not be applicable to all patient phenotypes, with potential for further sub-phenotyping. 


\section{Research Statement} 

The main research question of this capstone can be summarized as follows: \\
\textit{Is \SF a predictor of mortality in general ICU patient population? If yes, then over what range, and for which subsets of the population? 
}

By focusing on the SF ratio instead of only \Sp, we also investigate whether the former is a more helpful predictor as it theoretically allows us to account for the different levels of mechanical ventilation that an ICU patient receives. In essence, we believe it allows us to account for the patient's ability to convert inspired oxygen to peripheral oxygen saturation at the tissue level. 






