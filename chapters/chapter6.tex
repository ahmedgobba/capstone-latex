% Chapter Template

\chapter{Discussion} % Main chapter title

\label{Chapter6} % Change X to a consecutive number; for referencing this chapter elsewhere, use \ref{ChapterX}



\section{Conclusion} % Main chapter title


%----------------------------------------------------------------------------------------
%	SECTION 1
%----------------------------------------------------------------------------------------

To summarize our main results, we have found that average SF ratio over the first 24 hours of ICU stay (SF ratio) is significantly correlated with patient outcome, with the risk of  mortality significantly increasing at values of SF ratio below 180. Furthermore, the risk is exacerbated for patients on invasive mechanical ventilation than in those on non-invasive mechanical ventilation. 

We believe that this provides compelling reason to conduct further studies that examine our covariate of choice - average SF ratio over the first 24 hours of ICU stay - as a direct predictor of patient outcome. We hope that we have contributed to the field through our study. 


\section{Limitations}

%------------------------------------------------------------------------------

First of all, our study is an observational study (OS) based on retrospective data. As such, we do not control which patients get assigned what treatment methods, such as the type of mechanical ventilation. This is in contrast with a randomized controlled trial (RCT) where patients are randomly picked in order to ensure representation of the target population. In other words, in an OS, there are large observed and unobserved differences between different patient groups. This is called selection bias. Selection bias limits the generalizability of the data. 

Second, although we found significant correlation between SF ratio and mortality, we have not tested for causality. Our results, although promising, do not conclude that patient outcome is a causal effect of SF ratio. As such, the significant correlation we have established between SF ratio and mortality can only suggest for further studies that test for causality. These include the application of causal inference methods for time-varying exposure such as G-Computation to our data set \citep{snowden2011implementation} or setting up a randomized clinical trial. 
%-----------------------------------
%	SUBSECTION 2
%-----------------------------------
%
%\subsection{Subsection 2}
%Morbi rutrum odio eget arcu adipiscing sodales. Aenean et purus a est pulvinar pellentesque. Cras in elit neque, quis varius elit. Phasellus fringilla, nibh eu tempus venenatis, dolor elit posuere quam, quis adipiscing urna leo nec orci. Sed nec nulla auctor odio aliquet consequat. Ut nec nulla in ante ullamcorper aliquam at sed dolor. Phasellus fermentum magna in augue gravida cursus. Cras sed pretium lorem. Pellentesque eget ornare odio. Proin accumsan, massa viverra cursus pharetra, ipsum nisi lobortis velit, a malesuada dolor lorem eu neque.
%
%%----------------------------------------------------------------------------------------
%%	SECTION 2
%%----------------------------------------------------------------------------------------
%
%\section{Main Section 2}
%
%Sed ullamcorper quam eu nisl interdum at interdum enim egestas. Aliquam placerat justo sed lectus lobortis ut porta nisl porttitor. Vestibulum mi dolor, lacinia molestie gravida at, tempus vitae ligula. Donec eget quam sapien, in viverra eros. Donec pellentesque justo a massa fringilla non vestibulum metus vestibulum. Vestibulum in orci quis felis tempor lacinia. Vivamus ornare ultrices facilisis. Ut hendrerit volutpat vulputate. Morbi condimentum venenatis augue, id porta ipsum vulputate in. Curabitur luctus tempus justo. Vestibulum risus lectus, adipiscing nec condimentum quis, condimentum nec nisl. Aliquam dictum sagittis velit sed iaculis. Morbi tristique augue sit amet nulla pulvinar id facilisis ligula mollis. Nam elit libero, tincidunt ut aliquam at, molestie in quam. Aenean rhoncus vehicula hendrerit.