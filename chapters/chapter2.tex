% Chapter Template

\chapter{Data} % Main chapter title

\label{Chapter2} % Change X to a consecutive number; for referencing this chapter elsewhere, use \ref{ChapterX}

%----------------------------------------------------------------------------------------
%	SECTION 1
%----------------------------------------------------------------------------------------

\section{Data Overview}

For this capstone we used \textbf{MIMIC III}, an openly available relational database developed by the MIT Lab for Computational Physiology. It contains de-identified data of 61,532 intensive care unit stays: 53,432 stays for adult patients and 8,100 for neonatal patients at the Beth Israel Deaconess Medical Center over the June 2001 - October 2012. It includes demographics, vital signs, laboratory tests, medications, mortality, etc. The database is divided into different tables of data that contain information about a patient's stay and are linked to each via identifiers such as a unique hospital admission ID and a unique patient ID. 


\section{Data Extraction}

For the patient and stay identifiers we extract SUBJECT\_ID, HADM\_ID from \texttt{ADMISSIONS} table and ICUSTAY\_ID from \texttt{ICUSTAYS} table. From the \texttt{ADMISSIONS} table we extract the time of death of the patient if applicable and if it lies between the ICU admission time and ICU discharge time in \texttt{ICUSTAYS} table we indicate ICU mortality. Similarly if the time of death is between admission time and discharge time in the  \texttt{ADMISSIONS} table, we indicate Hospital mortality. From the \texttt{Patients} table we extract the gender of the patients' and calculate their age.   

For all patients and ICU stays we extract the \Fi values and their chart times from the \texttt{CHARTEVENTS} table. Keeping in mind that at normal atmospheric conditions, \Fi is around 21\%, we apply the following transformations. For the values between 0 and 1, we convert them to percentages by multiplying by a 100 and only keep those between 21\% and 100\%. Next, if the reading is recorded as greater than 1 but lower than 21,  the value is likely to be erroneous and we discard it. Next, if the value is between 21 and 100, the value is likely to already be a percentage and we take it as such. Finally, we discard all the values above a 100 that are remaining. From the same \texttt{CHARTEVENTS} table we extract the patients' height and weight. 

From the \texttt{CHARTEVENTS} table, we also extract \Sp values and chart times but we only keep those which indicate 0 for ERROR which stands for error in measurement.  Moreover, we filter the values and we discard those below 10 and above a 100 since they are either physiologically impossible or unlikely. 

At the end of this stage, the current dataset accounts for 46,476 Patients, 61,532 ICU Stays with 12,713,362 observations of either \Sp or \Fi or both. On further examination of the data, we find that for every \Fi measurement for a given ICU stay, for a given unique patient, there is a corresponding \Sp measurement at the same chart time but not vice versa. Accordingly, we restrict my data to only those chart times with both \Sp and \Fi measurements. This further subsets the number of observations further into 703,201 observations.h


%-----------------------------------
%	SUBSECTION 2
%-----------------------------------
%
%\subsection{Subsection 2}
%Morbi rutrum odio eget arcu adipiscing sodales. Aenean et purus a est pulvinar pellentesque. Cras in elit neque, quis varius elit. Phasellus fringilla, nibh eu tempus venenatis, dolor elit posuere quam, quis adipiscing urna leo nec orci. Sed nec nulla auctor odio aliquet consequat. Ut nec nulla in ante ullamcorper aliquam at sed dolor. Phasellus fermentum magna in augue gravida cursus. Cras sed pretium lorem. Pellentesque eget ornare odio. Proin accumsan, massa viverra cursus pharetra, ipsum nisi lobortis velit, a malesuada dolor lorem eu neque.
%
%%----------------------------------------------------------------------------------------
%%	SECTION 2
%%----------------------------------------------------------------------------------------
%
%\section{Main Section 2}
%
%Sed ullamcorper quam eu nisl interdum at interdum enim egestas. Aliquam placerat justo sed lectus lobortis ut porta nisl porttitor. Vestibulum mi dolor, lacinia molestie gravida at, tempus vitae ligula. Donec eget quam sapien, in viverra eros. Donec pellentesque justo a massa fringilla non vestibulum metus vestibulum. Vestibulum in orci quis felis tempor lacinia. Vivamus ornare ultrices facilisis. Ut hendrerit volutpat vulputate. Morbi condimentum venenatis augue, id porta ipsum vulputate in. Curabitur luctus tempus justo. Vestibulum risus lectus, adipiscing nec condimentum quis, condimentum nec nisl. Aliquam dictum sagittis velit sed iaculis. Morbi tristique augue sit amet nulla pulvinar id facilisis ligula mollis. Nam elit libero, tincidunt ut aliquam at, molestie in quam. Aenean rhoncus vehicula hendrerit.