% Appendix A

\chapter{Deleted Chapter} % Main appendix title

\label{AppendixA} % For referencing this appendix elsewhere, use \ref{AppendixA}

\subsection{Effect of Oxygen Therapy on Patients with Higher and Lower SF Ratio}

\subsubsection{Hypothesis}
Given that 180 seems to be the inflection point for SF ratio as seen in figure \ref{fig:results_general}, it is likely with patients with SF ratio < 180 require more respiratory support than those with SF ratio > 180. To be more precise,  we hypothesize the following:

\begin{enumerate}
	\item \textit{For patients with SF<180, the type of support within first 24 hours of ICU admission is important, with invasive mechanical ventilation having better outcomes compared to non-invasive mechanical ventilation.}
	
	\item \textit{For patients with SF>180, the type of support within first 24 hours of ICU admission would not differ in terms of outcomes.}
\end{enumerate}

\subsubsection{Test and Observations}
We test these hypotheses by splitting the patient cohort into 4 subpopulations based on these two splits as follows:
\begin{enumerate}
	\item Those with SF ratio greater than 180 and received non-invasive mechanical ventilation in the first 24 hours of ICU stay. 
	\item Those with SF ratio greater than 180 and received invasive mechanical ventilation in the first 24 hours of ICU stay. 
	\item Those with SF ratio less than 180 and received non-invasive mechanical ventilation in the first 24 hours of ICU stay.  
	\item Those with SF ratio less than 180 and received invasive mechanical ventilation in the first 24 hours of ICU stay.  
\end{enumerate}

We apply the model to each of the 4 subgroups and plot them below. 

