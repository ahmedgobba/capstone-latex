% Chapter Template

\chapter{Methods: How to model effect of Biomarkers on ICU mortality?} % Main chapter title

\label{Chapter3} % Change X to a consecutive number; for referencing this chapter elsewhere, use \ref{ChapterX}

\section{Analysis Goal}

To reiterate our goal, we not only want to test the significance of the correlation between SF ratio and patient outcome , we also, perhaps more importantly, want to examine over what ranges of SF ratio does significance hold if present. Moreover, if we do find a significant correlation, we want to examine whether such a relationship between SF ratio and patient outcome holds for various subsets of the population. Therefore, we need to find a method of modelling that allows us to examine for more than mere significance of the biomarker. 

\section{The Problem with Generalized Linear Models: They're Linear}

When we think of binary response variables such as mortality, we intuitively think of a linear logistic regression. The linear logistic regression  model belongs to a family of models called Generalized Linear Models (GLM). The word 'Linear' in the name does not stand for the relationship between the response variable and the predictor being a straight line, but rather to the fact that the predictor or a function of is is a modelled by a \textbf{linear} combination of the covariates. The general form of a GLM with $m$ covariates is: 

\begin{equation*}
g[\mathrm{E}(Y)]=\beta_{0}+\beta_{1} x_{1}+\ldots+\beta_{m} x_{m}
\end{equation*}

where $g$ is called the link function linking the expected value of $Y$ with a linear combination of the covariates, $
\beta_{0}+\beta_{1} x_{1}+\ldots+\beta_{m} x_{m}$ \citep{wood2017generalized}. That is, the model only allows for the response variable to be connected to the covariates in a linear manner. Since our goal is to explore an unknown relationship between mortality and SF ratio, we cannot introduce bias into our modelling by assuming this linear relationship. 


\section{General Additive Models}

\section{G-Computation}



